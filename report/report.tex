\title{60-371\\Artificial Intelligence\\Iterated Prisoner's Dilemma}
\author{
		Quinn Perfetto \\
        William Roeder \\
        David Valleau
}
\date{\today}

\documentclass[12pt]{article}

\usepackage{ amssymb }
\usepackage{longtable}
\usepackage[T1]{fontenc}
\usepackage{tocloft}
\renewcommand\cftsecleader{\cftdotfill{\cftdotsep}}


\begin{document}
\maketitle

\pagebreak
\tableofcontents
\pagebreak

\section{Abstract}

This paper explores the use of multiple Artificial Intelligence algorithms to
create optimal strategies to compete in the Iterated Prisoner's Dilemma.  The
study begins by examining the use of a genetic algorithm to breed the perfect
prisoner.  Several different configurations and fitness functions were tested and
the performance of each was contrasted.  The genetic algorithm was able to
consistently breed well performing prisoners.
A hill climbing approach was then taken,
which proved to be ineffective at producing a competitive strategy given the
lack of a natural successor function. Finally, machine learning... <INSERT
SUMMARY HERE>.

\pagebreak

\section{Introduction}

\pagebreak

\section{Genetic Algorithm}

\subsection{Representation}
\subsection{Initial Population}
\subsection{Fitness Function}
\subsection{Selection}
\subsection{Cross Over}
\subsection{Mutation}



\pagebreak

\section{Hill Climbing}

\pagebreak

\section{Machine Learning}

\end{document}
